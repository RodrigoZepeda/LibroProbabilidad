% Options for packages loaded elsewhere
\PassOptionsToPackage{unicode}{hyperref}
\PassOptionsToPackage{hyphens}{url}
%
\documentclass[
]{book}
\usepackage{lmodern}
\usepackage{amssymb,amsmath}
\usepackage{ifxetex,ifluatex}
\ifnum 0\ifxetex 1\fi\ifluatex 1\fi=0 % if pdftex
  \usepackage[T1]{fontenc}
  \usepackage[utf8]{inputenc}
  \usepackage{textcomp} % provide euro and other symbols
\else % if luatex or xetex
  \usepackage{unicode-math}
  \defaultfontfeatures{Scale=MatchLowercase}
  \defaultfontfeatures[\rmfamily]{Ligatures=TeX,Scale=1}
\fi
% Use upquote if available, for straight quotes in verbatim environments
\IfFileExists{upquote.sty}{\usepackage{upquote}}{}
\IfFileExists{microtype.sty}{% use microtype if available
  \usepackage[]{microtype}
  \UseMicrotypeSet[protrusion]{basicmath} % disable protrusion for tt fonts
}{}
\makeatletter
\@ifundefined{KOMAClassName}{% if non-KOMA class
  \IfFileExists{parskip.sty}{%
    \usepackage{parskip}
  }{% else
    \setlength{\parindent}{0pt}
    \setlength{\parskip}{6pt plus 2pt minus 1pt}}
}{% if KOMA class
  \KOMAoptions{parskip=half}}
\makeatother
\usepackage{xcolor}
\IfFileExists{xurl.sty}{\usepackage{xurl}}{} % add URL line breaks if available
\IfFileExists{bookmark.sty}{\usepackage{bookmark}}{\usepackage{hyperref}}
\hypersetup{
  pdftitle={Notas para un curso de probabilidad},
  pdfauthor={Rodrigo Zepeda-Tello / Luis Carlos Bernal},
  hidelinks,
  pdfcreator={LaTeX via pandoc}}
\urlstyle{same} % disable monospaced font for URLs
\usepackage{longtable,booktabs}
% Correct order of tables after \paragraph or \subparagraph
\usepackage{etoolbox}
\makeatletter
\patchcmd\longtable{\par}{\if@noskipsec\mbox{}\fi\par}{}{}
\makeatother
% Allow footnotes in longtable head/foot
\IfFileExists{footnotehyper.sty}{\usepackage{footnotehyper}}{\usepackage{footnote}}
\makesavenoteenv{longtable}
\usepackage{graphicx,grffile}
\makeatletter
\def\maxwidth{\ifdim\Gin@nat@width>\linewidth\linewidth\else\Gin@nat@width\fi}
\def\maxheight{\ifdim\Gin@nat@height>\textheight\textheight\else\Gin@nat@height\fi}
\makeatother
% Scale images if necessary, so that they will not overflow the page
% margins by default, and it is still possible to overwrite the defaults
% using explicit options in \includegraphics[width, height, ...]{}
\setkeys{Gin}{width=\maxwidth,height=\maxheight,keepaspectratio}
% Set default figure placement to htbp
\makeatletter
\def\fps@figure{htbp}
\makeatother
\setlength{\emergencystretch}{3em} % prevent overfull lines
\providecommand{\tightlist}{%
  \setlength{\itemsep}{0pt}\setlength{\parskip}{0pt}}
\setcounter{secnumdepth}{5}
\usepackage{booktabs}
\usepackage[]{natbib}
\bibliographystyle{apalike}

\title{Notas para un curso de probabilidad}
\author{Rodrigo Zepeda-Tello / Luis Carlos Bernal}
\date{2020-12-18}

\begin{document}
\maketitle

{
\setcounter{tocdepth}{1}
\tableofcontents
}
\hypertarget{algunos-problemas-probabiluxedsticos}{%
\chapter{Algunos problemas probabilísticos}\label{algunos-problemas-probabiluxedsticos}}

\hypertarget{repaso_diferencial}{%
\chapter{Repaso de Cálculo Diferencial}\label{repaso_diferencial}}

\hypertarget{ejercicios-que-se-espera-puedas-resolver}{%
\section{Ejercicios que se espera puedas resolver}\label{ejercicios-que-se-espera-puedas-resolver}}

\hypertarget{errores-comunes-de-uxe1lgebra}{%
\section{Errores comunes de álgebra}\label{errores-comunes-de-uxe1lgebra}}

\hypertarget{propiedades-de-los-logaritmos-y-la-exponencial}{%
\section{Propiedades de los logaritmos y la exponencial}\label{propiedades-de-los-logaritmos-y-la-exponencial}}

\hypertarget{derivadas}{%
\section{Derivadas}\label{derivadas}}

\hypertarget{definiciuxf3n-como-luxedmite}{%
\subsection{Definición como límite}\label{definiciuxf3n-como-luxedmite}}

\hypertarget{derivada-de-la-suma}{%
\subsection{Derivada de la suma}\label{derivada-de-la-suma}}

\hypertarget{derivada-del-producto}{%
\subsection{Derivada del producto}\label{derivada-del-producto}}

\hypertarget{derivada-de-una-divisiuxf3n}{%
\subsection{Derivada de una división}\label{derivada-de-una-divisiuxf3n}}

\hypertarget{regla-de-la-cadena}{%
\subsection{Regla de la cadena}\label{regla-de-la-cadena}}

\hypertarget{derivaciuxf3n-logaruxedtmica}{%
\subsection{Derivación logarítmica}\label{derivaciuxf3n-logaruxedtmica}}

\hypertarget{derivadas-parciales}{%
\subsection{Derivadas parciales}\label{derivadas-parciales}}

\hypertarget{gradiente}{%
\subsection{Gradiente}\label{gradiente}}

\hypertarget{matriz-hessiana}{%
\subsection{Matriz hessiana}\label{matriz-hessiana}}

\hypertarget{intro}{%
\chapter{Introducción}\label{intro}}

\hypertarget{algunos-problemas-de-probabilidad}{%
\section{Algunos problemas de probabilidad}\label{algunos-problemas-de-probabilidad}}

\begin{enumerate}
\def\labelenumi{\arabic{enumi}.}
\item
  Un dado de 6 caras.
\item
  Un dado de 12 caras.
\item
  Dos dados de 6 caras cada uno.
\item
  Ejercicio: dos dados de 12 caras cada uno.
\item
\end{enumerate}

Definición preliminar: proba como eventos favorables / eventos totales
Anotación de por qué eso no funciona siempre.

\hypertarget{ejemplos-sencillos-de-casos-favorables-entre-totales}{%
\subsection{Ejemplos sencillos de casos favorables entre totales}\label{ejemplos-sencillos-de-casos-favorables-entre-totales}}

\hypertarget{teoruxeda-buxe1sica-de-conjuntos}{%
\subsection{Teoría básica de conjuntos}\label{teoruxeda-buxe1sica-de-conjuntos}}

\hypertarget{axiomas-de-kolmogorov-versiuxf3n-1}{%
\subsection{Axiomas de Kolmogorov (versión 1)}\label{axiomas-de-kolmogorov-versiuxf3n-1}}

\hypertarget{propiedades-de-la-probabilidad}{%
\subsection{Propiedades de la probabilidad}\label{propiedades-de-la-probabilidad}}

\hypertarget{ejercicios-resueltos}{%
\section{Ejercicios resueltos}\label{ejercicios-resueltos}}

\hypertarget{conteo}{%
\chapter{Conteo}\label{conteo}}

\hypertarget{principios-de-conteo}{%
\section{Principios de conteo}\label{principios-de-conteo}}

\hypertarget{permutaciones-muestras-ordenadas}{%
\subsection{Permutaciones (muestras ordenadas)}\label{permutaciones-muestras-ordenadas}}

\hypertarget{combinaciones-subpoblaciones}{%
\subsection{Combinaciones (subpoblaciones)}\label{combinaciones-subpoblaciones}}

\hypertarget{particiones-problemas-de-ocupaciuxf3n}{%
\subsection{Particiones (problemas de ocupación)}\label{particiones-problemas-de-ocupaciuxf3n}}

\hypertarget{representaciuxf3n-mediante-uxe1rboles}{%
\subsection{Representación mediante árboles}\label{representaciuxf3n-mediante-uxe1rboles}}

\hypertarget{representaciuxf3n-mediante-puntos-y-luxedneas}{%
\subsection{Representación mediante puntos y líneas}\label{representaciuxf3n-mediante-puntos-y-luxedneas}}

\hypertarget{principio-de-inclusiuxf3nexclusiuxf3n}{%
\subsection{Principio de inclusión/exclusión}\label{principio-de-inclusiuxf3nexclusiuxf3n}}

\hypertarget{conteo-con-proba}{%
\section{Conteo con proba}\label{conteo-con-proba}}

\hypertarget{muestreo-aleatorio-simple-con-reemplazo}{%
\subsection{Muestreo aleatorio simple con reemplazo}\label{muestreo-aleatorio-simple-con-reemplazo}}

\hypertarget{muestreo-aleatorio-simple-sin-reemplazo}{%
\subsection{Muestreo aleatorio simple sin reemplazo}\label{muestreo-aleatorio-simple-sin-reemplazo}}

\hypertarget{uxe1rboles-binomiales}{%
\subsection{Árboles binomiales}\label{uxe1rboles-binomiales}}

\hypertarget{urnas-de-polya}{%
\subsection{Urnas de Polya}\label{urnas-de-polya}}

\hypertarget{distribuciuxf3n-binomial}{%
\subsection{Distribución Binomial}\label{distribuciuxf3n-binomial}}

\hypertarget{distribuciuxf3n-hipergeomuxe9trica}{%
\subsection{Distribución Hipergeométrica}\label{distribuciuxf3n-hipergeomuxe9trica}}

\hypertarget{ejercicios-resueltos-1}{%
\section{Ejercicios resueltos}\label{ejercicios-resueltos-1}}

\hypertarget{probacond}{%
\chapter{Probabilidad condicional e independencia}\label{probacond}}

\hypertarget{ejemplos-de-proyecciuxf3n-sobre-el-espacio-muestral}{%
\section{Ejemplos de proyección sobre el espacio muestral}\label{ejemplos-de-proyecciuxf3n-sobre-el-espacio-muestral}}

\hypertarget{definiciuxf3n-y-propiedades-de-probabilidad-condicional}{%
\section{Definición y propiedades de probabilidad condicional}\label{definiciuxf3n-y-propiedades-de-probabilidad-condicional}}

\hypertarget{independencia-de-eventos}{%
\section{Independencia de eventos}\label{independencia-de-eventos}}

\hypertarget{regla-de-la-multiplicaciuxf3n}{%
\section{Regla de la multiplicación}\label{regla-de-la-multiplicaciuxf3n}}

\hypertarget{probabilidad-total}{%
\section{Probabilidad total}\label{probabilidad-total}}

\hypertarget{teorema-de-bayes}{%
\section{Teorema de Bayes}\label{teorema-de-bayes}}

\hypertarget{ejemplo-en-probabilidad-de-covid-condicional-a-muestra-positiva}{%
\subsection{Ejemplo en probabilidad de covid condicional a muestra positiva}\label{ejemplo-en-probabilidad-de-covid-condicional-a-muestra-positiva}}

\hypertarget{ejemplo-en-muestreo-secuencial}{%
\subsection{Ejemplo en muestreo secuencial}\label{ejemplo-en-muestreo-secuencial}}

\hypertarget{ejemplo-de-la-paradoja-de-simpson}{%
\subsection{Ejemplo de la paradoja de Simpson}\label{ejemplo-de-la-paradoja-de-simpson}}

\hypertarget{ejercicios-resueltos-2}{%
\section{Ejercicios resueltos}\label{ejercicios-resueltos-2}}

  \bibliography{book.bib,packages.bib}

\end{document}
