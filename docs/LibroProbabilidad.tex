% Options for packages loaded elsewhere
\PassOptionsToPackage{unicode}{hyperref}
\PassOptionsToPackage{hyphens}{url}
%
\documentclass[
]{book}
\usepackage{lmodern}
\usepackage{amssymb,amsmath}
\usepackage{ifxetex,ifluatex}
\ifnum 0\ifxetex 1\fi\ifluatex 1\fi=0 % if pdftex
  \usepackage[T1]{fontenc}
  \usepackage[utf8]{inputenc}
  \usepackage{textcomp} % provide euro and other symbols
\else % if luatex or xetex
  \usepackage{unicode-math}
  \defaultfontfeatures{Scale=MatchLowercase}
  \defaultfontfeatures[\rmfamily]{Ligatures=TeX,Scale=1}
\fi
% Use upquote if available, for straight quotes in verbatim environments
\IfFileExists{upquote.sty}{\usepackage{upquote}}{}
\IfFileExists{microtype.sty}{% use microtype if available
  \usepackage[]{microtype}
  \UseMicrotypeSet[protrusion]{basicmath} % disable protrusion for tt fonts
}{}
\makeatletter
\@ifundefined{KOMAClassName}{% if non-KOMA class
  \IfFileExists{parskip.sty}{%
    \usepackage{parskip}
  }{% else
    \setlength{\parindent}{0pt}
    \setlength{\parskip}{6pt plus 2pt minus 1pt}}
}{% if KOMA class
  \KOMAoptions{parskip=half}}
\makeatother
\usepackage{xcolor}
\IfFileExists{xurl.sty}{\usepackage{xurl}}{} % add URL line breaks if available
\IfFileExists{bookmark.sty}{\usepackage{bookmark}}{\usepackage{hyperref}}
\hypersetup{
  pdftitle={Notas para un curso de probabilidad},
  pdfauthor={Rodrigo Zepeda-Tello / Luis Carlos Bernal},
  hidelinks,
  pdfcreator={LaTeX via pandoc}}
\urlstyle{same} % disable monospaced font for URLs
\usepackage{longtable,booktabs}
% Correct order of tables after \paragraph or \subparagraph
\usepackage{etoolbox}
\makeatletter
\patchcmd\longtable{\par}{\if@noskipsec\mbox{}\fi\par}{}{}
\makeatother
% Allow footnotes in longtable head/foot
\IfFileExists{footnotehyper.sty}{\usepackage{footnotehyper}}{\usepackage{footnote}}
\makesavenoteenv{longtable}
\usepackage{graphicx,grffile}
\makeatletter
\def\maxwidth{\ifdim\Gin@nat@width>\linewidth\linewidth\else\Gin@nat@width\fi}
\def\maxheight{\ifdim\Gin@nat@height>\textheight\textheight\else\Gin@nat@height\fi}
\makeatother
% Scale images if necessary, so that they will not overflow the page
% margins by default, and it is still possible to overwrite the defaults
% using explicit options in \includegraphics[width, height, ...]{}
\setkeys{Gin}{width=\maxwidth,height=\maxheight,keepaspectratio}
% Set default figure placement to htbp
\makeatletter
\def\fps@figure{htbp}
\makeatother
\setlength{\emergencystretch}{3em} % prevent overfull lines
\providecommand{\tightlist}{%
  \setlength{\itemsep}{0pt}\setlength{\parskip}{0pt}}
\setcounter{secnumdepth}{5}
\usepackage{booktabs}
\usepackage[]{natbib}
\bibliographystyle{apalike}

\title{Notas para un curso de probabilidad}
\author{Rodrigo Zepeda-Tello / Luis Carlos Bernal}
\date{2020-12-19}

\begin{document}
\maketitle

{
\setcounter{tocdepth}{1}
\tableofcontents
}
\hypertarget{algunos-problemas-probabiluxedsticos}{%
\chapter{Algunos problemas probabilísticos}\label{algunos-problemas-probabiluxedsticos}}

\hypertarget{repaso_diferencial}{%
\chapter{Repaso de Cálculo Diferencial}\label{repaso_diferencial}}

\hypertarget{ejercicios-que-se-espera-puedas-resolver}{%
\section{Ejercicios que se espera puedas resolver}\label{ejercicios-que-se-espera-puedas-resolver}}

\hypertarget{errores-comunes-de-uxe1lgebra}{%
\section{Errores comunes de álgebra}\label{errores-comunes-de-uxe1lgebra}}

\hypertarget{propiedades-de-los-logaritmos-y-la-exponencial}{%
\section{Propiedades de los logaritmos y la exponencial}\label{propiedades-de-los-logaritmos-y-la-exponencial}}

\hypertarget{derivadas}{%
\section{Derivadas}\label{derivadas}}

\hypertarget{definiciuxf3n-como-luxedmite}{%
\subsection{Definición como límite}\label{definiciuxf3n-como-luxedmite}}

\hypertarget{derivada-de-la-suma}{%
\subsection{Derivada de la suma}\label{derivada-de-la-suma}}

\hypertarget{derivada-del-producto}{%
\subsection{Derivada del producto}\label{derivada-del-producto}}

\hypertarget{derivada-de-una-divisiuxf3n}{%
\subsection{Derivada de una división}\label{derivada-de-una-divisiuxf3n}}

\hypertarget{regla-de-la-cadena}{%
\subsection{Regla de la cadena}\label{regla-de-la-cadena}}

\hypertarget{derivaciuxf3n-logaruxedtmica}{%
\subsection{Derivación logarítmica}\label{derivaciuxf3n-logaruxedtmica}}

\hypertarget{derivadas-parciales}{%
\subsection{Derivadas parciales}\label{derivadas-parciales}}

\hypertarget{gradiente}{%
\subsection{Gradiente}\label{gradiente}}

\hypertarget{matriz-hessiana}{%
\subsection{Matriz hessiana}\label{matriz-hessiana}}

\hypertarget{intro}{%
\chapter{Introducción}\label{intro}}

Probabilidad es una rama ``nueva'' de matemáticas (¡tiene menos de 100 años!) que sólo entró a ser considerada como tal después de que Kolmogorov \href{https://link.springer.com/article/10.1007/BF00348144}{estableciera sus axiomas en 1933}. Inicialmente (y la aproximación que tomamos aquí es histórica) probabilidad era sólo usada para juegos de azar tomando estimaciones como la siguiente:
\[
\text{Probabilidad de ganar} = \dfrac{\text{Casos favorables}}{\text{Todos los casos posibles}}.
\]
Por ejemplo, en una apuesta por el lado resultante de un tiro de moneda con posibles resultados \emph{cara} ó \emph{cruz}, al elegir \emph{cruz} se contempla una probabilidad de ganar de \(1/2\) pues los \(\text{Casos favorables}\) son sólo uno (que salga \emph{cruz}) y \(\text{Todos los casos posibles}\) (descontando cosas raras como que la moneda caiga de lado o que el mundo se acabe antes de que la moneda toque el piso) son dos: caer \emph{cara} ó \emph{cruz}. Esto lleva al siguiente cálculo de fracciones:

\[
\text{Probabilidad de ganar} = \dfrac{\text{Casos favorables}}{\text{Todos los casos}} = \dfrac{1}{2}
\]
En el caso del tiro de la moneda la aproximación inicial, referente a una moneda \emph{justa}, de \(1/2\) de probabilidad de ganar puede no mantenerse siempre. En monedas \emph{reales} se sabe que no es exactamente \(1/2\) la probabilidad (\href{https://statweb.stanford.edu/~susan/papers/headswithJ.pdf}{el relieve sí importa} de forma minúscula además de que \href{https://www.ncbi.nlm.nih.gov/pmc/articles/PMC2789164/}{hay gente que puede manipular que una moneda caiga con mayor probabilidad de la manera deseada}). Podríamos entonces tener el caso de una moneda cargada donde después de arrojarla 100 veces 99 de ellas resultaron en \emph{cara}. Si quisiéramos adivinar el resultado del siguiente tiro (el \(101\)) no sería tan buena aproximación el contar
\[
\dfrac{\text{Casos favorables}}{\text{Todos los casos}}
\]
pues sigue siendo \(1\) caso en el que ganas (que caiga \emph{cara}) de dos casos (que caiga \emph{cara} o \emph{cruz}). Una mejor aproximación al tiro de la moneda sería (en este caso) contar toda la información de las frecuencias históricas para considerar los \(100\) tiros que ya conocemos:
\[
\text{Probabilidad de ganar} =  \dfrac{\text{Casos en los que salió cruz}}{\text{Todos los tiros pasados}} = \dfrac{1}{100}
\]
En este caso de una moneda \emph{cargada} (es decir que está manipulada para caer más de un lado que de otro) las probabilidades resultan de un conteo de frecuencias pasadas de eventos que ya ocurrieron y no necesariamente resultan en eventos de misma probabilidad (llamados equiprobables). Específicamente en este caso la probabilidad de ganar si elijes cruz es \(1/100\) mientras que la de perder es \(99/100\). Este es un ejemplo de que \textbf{no siempre por tener dos posibilidades la probabilidad de una es \(1/2\) y, por supuesto que, de manera general, no por tener \(n\) probabilidades la probabilidad de una de ellas es \(1/n\).}

A la anterior interpretación de la probabilidad que se dedica a contar del histórico cuántos eventos de cada tipo han ocurrido y dividirlo entre los posibles se le conoce como \emph{frecuentista}. Sin embargo dicha interpretación es incompleta. Bajo la idea de\\
\[
\dfrac{\text{Casos favorables}}{\text{Todos los casos}}
\]
o bien de
\[
\dfrac{\text{Veces en las que ha ocurrido lo que me interesa}}{\text{Todos los casos}}
\]
no se pueden plantear probabilidades para eventos futuros. Por ejemplo: ¿cuál es la probabilidad de éxito de un nuevo producto en el mercado? ¿qué candidato es más probable que triunfe en las próximas elecciones? ¿cuál es la probabilidad de que mañana llueva en mi ciudad? Todos estos eventos no pueden modelarse contando el número de veces que ha ocurrido el evento en el pasado ¡porque no son eventos que se repitan! Tampoco tiene mucho sentido modelarlo considerando casos favorables entre casos totales: aunque un producto puede tener éxito o no tenerlo la probabilidad no es 50 y 50 (los mercadólogos se aseguran de ello); en una elección con tres candidatos no ocurre que cada uno tenga 33.33\% de probabilidad de ganar (siempre hay unos con mayores posibilidades); si mañana llueve o no depende de muchos factores desde la presión barométrica, el clima histórico de mi ciudad, la temperatura (bien sabes que hay climas más probables que otros\footnote{O eres de Monterrey donde cualquier evento climático puede ocurrir en un abrir y cerrar de ojos.}). Para esto último (ya llegaremos) se requerirá maquinaria más pesada: el invento de un concepto conocido como \emph{variables aleatorias} y \emph{el teorema de Bayes} que en conjunto permiten integrar información adicional (como presión barométrica en el caso de la lluvia o datos de productos similares en el caso del mercado) para poder modelar la probabilidad.

Si la probabilidad no es sólo conteo de eventos (casos favorables y totales) ¿qué es entonces? En esto se abre una discusión eterna. Hay probabilistas que afirman que es una medida de creencias (si yo te digo que hay una probabilidad del 70\% de que llueva mañana es porque \emph{creo} que la lluvia del día de mañana es un evento posible y \emph{elijo} cuantificar mi creencia con ese número). A estos probabilistas se les conoce como Bayesianos. Por otro lado, otros probabilistas afirman que es una extensión de la lógica. Al enunciado \emph{Mañana va a llover} no se le puede asignar una categoría de \emph{verdad} o \emph{falsedad} el día de hoy. Empero, se puede asignar una probabilidad (en este caso el 70\%) que cuantifique qué tan verdadero es un enunciado. De esta forma, la probabilidad extiende la lógica diciendo qué tan cierto es algo con la probabilidad de 0\% siendo un evento que es completamente falso y 100\% un evento completamente verdadero.

Yo, Rodrigo, soy partidario de esta última interpretación. Sin embargo, para fines de estas notas, todas las interpretaciones, sean conteos de frecuencia, grados de credibilidad o grados de verdad, van a ser equivalentes y sólo en detalles minúsculos notaremos algunas diferencias interpretativas.

Por ahora, comenzaremos nuestra aventura por el mundo de la probabilidad analizando la probabilidad clásica: el conteo de cuántos eventos favorables hay sobre el total de eventos posibles.

\begin{quote}
En esta sección estudiaremos: algunos problemas de probabilidad basado en conteos muy simples (equiprobables) y más adelante nos enfocaremos sólo en contar eventos (cuántas formas hay de que ocurra algo). En la siguiente sección utilizaremos estos conteos para comenzar a hablar de probabilidades
\end{quote}

\hypertarget{algunos-problemas-simples-de-probabilidad}{%
\section{Algunos problemas ``simples'' de probabilidad}\label{algunos-problemas-simples-de-probabilidad}}

\begin{enumerate}
\def\labelenumi{\arabic{enumi}.}
\tightlist
\item
  Un dado de 6 caras.
\item
  Un dado de 12 caras.
\item
  Dos dados de 6 caras cada uno.
\item
  Ejercicio: dos dados de 12 caras cada uno.
\item
  Ejercicio: dos dados de 12 caras cada uno condicional a la cara de un dado
\end{enumerate}

Definición preliminar: proba como eventos favorables / eventos totales
Anotación de por qué eso no funciona siempre.

\hypertarget{ejemplos-sencillos-de-casos-favorables-entre-totales}{%
\subsection{Ejemplos sencillos de casos favorables entre totales}\label{ejemplos-sencillos-de-casos-favorables-entre-totales}}

\hypertarget{principios-de-conteo}{%
\section{Principios de conteo}\label{principios-de-conteo}}

\hypertarget{permutaciones-muestras-ordenadas}{%
\subsection{Permutaciones (muestras ordenadas)}\label{permutaciones-muestras-ordenadas}}

\hypertarget{combinaciones-subpoblaciones}{%
\subsection{Combinaciones (subpoblaciones)}\label{combinaciones-subpoblaciones}}

\hypertarget{particiones-problemas-de-ocupaciuxf3n}{%
\subsection{Particiones (problemas de ocupación)}\label{particiones-problemas-de-ocupaciuxf3n}}

\hypertarget{representaciuxf3n-mediante-uxe1rboles}{%
\subsection{Representación mediante árboles}\label{representaciuxf3n-mediante-uxe1rboles}}

\hypertarget{representaciuxf3n-mediante-puntos-y-luxedneas}{%
\subsection{Representación mediante puntos y líneas}\label{representaciuxf3n-mediante-puntos-y-luxedneas}}

\hypertarget{principio-de-inclusiuxf3nexclusiuxf3n}{%
\subsection{Principio de inclusión/exclusión}\label{principio-de-inclusiuxf3nexclusiuxf3n}}

\hypertarget{ejercicios-resueltos}{%
\section{Ejercicios resueltos}\label{ejercicios-resueltos}}

\hypertarget{conteo}{%
\chapter{Conteo}\label{conteo}}

\hypertarget{conteo-con-proba}{%
\section{Conteo con proba}\label{conteo-con-proba}}

\hypertarget{muestreo-aleatorio-simple-con-reemplazo}{%
\subsection{Muestreo aleatorio simple con reemplazo}\label{muestreo-aleatorio-simple-con-reemplazo}}

\hypertarget{muestreo-aleatorio-simple-sin-reemplazo}{%
\subsection{Muestreo aleatorio simple sin reemplazo}\label{muestreo-aleatorio-simple-sin-reemplazo}}

\hypertarget{uxe1rboles-binomiales}{%
\subsection{Árboles binomiales}\label{uxe1rboles-binomiales}}

\hypertarget{urnas-de-polya}{%
\subsection{Urnas de Polya}\label{urnas-de-polya}}

\hypertarget{distribuciuxf3n-binomial}{%
\subsection{Distribución Binomial}\label{distribuciuxf3n-binomial}}

\hypertarget{distribuciuxf3n-hipergeomuxe9trica}{%
\subsection{Distribución Hipergeométrica}\label{distribuciuxf3n-hipergeomuxe9trica}}

\hypertarget{teoruxeda-buxe1sica-de-conjuntos}{%
\subsection{Teoría básica de conjuntos}\label{teoruxeda-buxe1sica-de-conjuntos}}

\hypertarget{axiomas-de-kolmogorov-versiuxf3n-1}{%
\subsection{Axiomas de Kolmogorov (versión 1)}\label{axiomas-de-kolmogorov-versiuxf3n-1}}

\hypertarget{propiedades-de-la-probabilidad}{%
\subsection{Propiedades de la probabilidad}\label{propiedades-de-la-probabilidad}}

\hypertarget{principio-de-inclusiuxf3nexclusiuxf3n-1}{%
\subsection{Principio de inclusión/exclusión}\label{principio-de-inclusiuxf3nexclusiuxf3n-1}}

\hypertarget{ejercicios-resueltos-1}{%
\section{Ejercicios resueltos}\label{ejercicios-resueltos-1}}

\hypertarget{probacond}{%
\chapter{Probabilidad condicional e independencia}\label{probacond}}

\hypertarget{ejemplos-de-proyecciuxf3n-sobre-el-espacio-muestral}{%
\section{Ejemplos de proyección sobre el espacio muestral}\label{ejemplos-de-proyecciuxf3n-sobre-el-espacio-muestral}}

\hypertarget{definiciuxf3n-y-propiedades-de-probabilidad-condicional}{%
\section{Definición y propiedades de probabilidad condicional}\label{definiciuxf3n-y-propiedades-de-probabilidad-condicional}}

\hypertarget{independencia-de-eventos}{%
\section{Independencia de eventos}\label{independencia-de-eventos}}

\hypertarget{regla-de-la-multiplicaciuxf3n}{%
\section{Regla de la multiplicación}\label{regla-de-la-multiplicaciuxf3n}}

\hypertarget{probabilidad-total}{%
\section{Probabilidad total}\label{probabilidad-total}}

\hypertarget{teorema-de-bayes}{%
\section{Teorema de Bayes}\label{teorema-de-bayes}}

\hypertarget{ejemplo-en-probabilidad-de-covid-condicional-a-muestra-positiva}{%
\subsection{Ejemplo en probabilidad de covid condicional a muestra positiva}\label{ejemplo-en-probabilidad-de-covid-condicional-a-muestra-positiva}}

\hypertarget{ejemplo-en-muestreo-secuencial}{%
\subsection{Ejemplo en muestreo secuencial}\label{ejemplo-en-muestreo-secuencial}}

\hypertarget{ejemplo-de-la-paradoja-de-simpson}{%
\subsection{Ejemplo de la paradoja de Simpson}\label{ejemplo-de-la-paradoja-de-simpson}}

\hypertarget{ejercicios-resueltos-2}{%
\section{Ejercicios resueltos}\label{ejercicios-resueltos-2}}

  \bibliography{book.bib,packages.bib}

\end{document}
